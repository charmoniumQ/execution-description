%%%%%%%%%%%%%%%%%%%%
% This file is automatically generated. Refer to the markdown source.
%%%%%%%%%%%%%%%%%%%%

\documentclass[manuscript,authordraft]{acmart}
\usepackage[utf8]{inputenc}
% ACM options first in the order they appear in acmguide.pdf


  \acmConference[]{Soft. Eng. 4 Res. Sci.}{July 23--27, 2023}{Portland,
OR}

\title{Wanted: standards for automatic reproducibility of computational
experiments}

\author{Samuel Grayson}
  
  \email{grayson5@illinois.edu}
  \orcid{0000-0001-5411-356X}
      \affiliation{%
    %
    \institution{University of Illinois Urbana-Champaign}%
    \department{Department of Computer Science}%
    \streetaddress{201 North Goodwin Avenue MC 258}%
    \city{Urbana}%
    \state{IL}%
    \postcode{61801-2302}%
    \country{USA}%
    }
    
  
\author{Reed Milewicz}
  
  \email{rmilewi@sandia.gov}
  \orcid{0000-0002-1701-0008}
      \affiliation{%
    %
    \institution{Sandia National Laboratories}%
    \department{Software Engineering and Research Department}%
    \streetaddress{1515 Eubank Blvd SE1515 Eubank Blvd SE}%
    \city{Albuquerque}%
    \state{NM}%
    \postcode{87123}%
    \country{USA}%
    }
    
  
\author{Joshua Teves}
  
  \email{jbteves@sandia.gov}
  \orcid{0000-0002-7767-0067}
      \affiliation{%
    %
    \institution{Sandia National Laboratories}%
    \department{Software Engineering and Research Department}%
    \streetaddress{1515 Eubank Blvd SE1515 Eubank Blvd SE}%
    \city{Albuquerque}%
    \state{NM}%
    \postcode{87123}%
    \country{USA}%
    }
    
  
\author{Daniel S. Katz}
  
  \email{dskatz@illinois.edu}
  \orcid{0000-0001-5934-7525}
      \affiliation{%
    %
    \institution{University of Illinois Urbana-Champaign Department of
Computer Science}%
    \department{Department of Computer Science}%
    \streetaddress{201 North Goodwin Avenue MC 258}%
    \city{Urbana}%
    \state{IL}%
    \postcode{61801-2302}%
    \country{USA}%
    }
    
  
\author{Darko Marinov}
  
  \email{marinov@illinois.edu}
  \orcid{0000-0001-5023-3492}
      \affiliation{%
    %
    \institution{University of Illinois Urbana-Champaign}%
    \department{Department of Computer Science}%
    \streetaddress{201 North Goodwin Avenue MC 258}%
    \city{Urbana}%
    \state{IL}%
    \postcode{61801-2302}%
    \country{USA}%
    }
    
  



\acmYear{2023}









\setcopyright{none}


\settopmatter{printacmref=false}



% \RequirePackage[
%   datamodel=acmdatamodel,
%   style=acmnumeric, % use style=acmauthoryear for publications that require it
% ]{biblatex}
% % \addbibresource{sams-zotero-export.bib}
% % \addbibresource{manual.bib}
% 
\usepackage{hyperref}



\usepackage{color}
\usepackage{fancyvrb}
\newcommand{\VerbBar}{|}
\newcommand{\VERB}{\Verb[commandchars=\\\{\}]}
\DefineVerbatimEnvironment{Highlighting}{Verbatim}{commandchars=\\\{\}}
% Add ',fontsize=\small' for more characters per line
\newenvironment{Shaded}{}{}
\newcommand{\AlertTok}[1]{\textcolor[rgb]{1.00,0.00,0.00}{\textbf{#1}}}
\newcommand{\AnnotationTok}[1]{\textcolor[rgb]{0.38,0.63,0.69}{\textbf{\textit{#1}}}}
\newcommand{\AttributeTok}[1]{\textcolor[rgb]{0.49,0.56,0.16}{#1}}
\newcommand{\BaseNTok}[1]{\textcolor[rgb]{0.25,0.63,0.44}{#1}}
\newcommand{\BuiltInTok}[1]{\textcolor[rgb]{0.00,0.50,0.00}{#1}}
\newcommand{\CharTok}[1]{\textcolor[rgb]{0.25,0.44,0.63}{#1}}
\newcommand{\CommentTok}[1]{\textcolor[rgb]{0.38,0.63,0.69}{\textit{#1}}}
\newcommand{\CommentVarTok}[1]{\textcolor[rgb]{0.38,0.63,0.69}{\textbf{\textit{#1}}}}
\newcommand{\ConstantTok}[1]{\textcolor[rgb]{0.53,0.00,0.00}{#1}}
\newcommand{\ControlFlowTok}[1]{\textcolor[rgb]{0.00,0.44,0.13}{\textbf{#1}}}
\newcommand{\DataTypeTok}[1]{\textcolor[rgb]{0.56,0.13,0.00}{#1}}
\newcommand{\DecValTok}[1]{\textcolor[rgb]{0.25,0.63,0.44}{#1}}
\newcommand{\DocumentationTok}[1]{\textcolor[rgb]{0.73,0.13,0.13}{\textit{#1}}}
\newcommand{\ErrorTok}[1]{\textcolor[rgb]{1.00,0.00,0.00}{\textbf{#1}}}
\newcommand{\ExtensionTok}[1]{#1}
\newcommand{\FloatTok}[1]{\textcolor[rgb]{0.25,0.63,0.44}{#1}}
\newcommand{\FunctionTok}[1]{\textcolor[rgb]{0.02,0.16,0.49}{#1}}
\newcommand{\ImportTok}[1]{\textcolor[rgb]{0.00,0.50,0.00}{\textbf{#1}}}
\newcommand{\InformationTok}[1]{\textcolor[rgb]{0.38,0.63,0.69}{\textbf{\textit{#1}}}}
\newcommand{\KeywordTok}[1]{\textcolor[rgb]{0.00,0.44,0.13}{\textbf{#1}}}
\newcommand{\NormalTok}[1]{#1}
\newcommand{\OperatorTok}[1]{\textcolor[rgb]{0.40,0.40,0.40}{#1}}
\newcommand{\OtherTok}[1]{\textcolor[rgb]{0.00,0.44,0.13}{#1}}
\newcommand{\PreprocessorTok}[1]{\textcolor[rgb]{0.74,0.48,0.00}{#1}}
\newcommand{\RegionMarkerTok}[1]{#1}
\newcommand{\SpecialCharTok}[1]{\textcolor[rgb]{0.25,0.44,0.63}{#1}}
\newcommand{\SpecialStringTok}[1]{\textcolor[rgb]{0.73,0.40,0.53}{#1}}
\newcommand{\StringTok}[1]{\textcolor[rgb]{0.25,0.44,0.63}{#1}}
\newcommand{\VariableTok}[1]{\textcolor[rgb]{0.10,0.09,0.49}{#1}}
\newcommand{\VerbatimStringTok}[1]{\textcolor[rgb]{0.25,0.44,0.63}{#1}}
\newcommand{\WarningTok}[1]{\textcolor[rgb]{0.38,0.63,0.69}{\textbf{\textit{#1}}}}


\usepackage{microtype}





\setlength{\emergencystretch}{3em}  % prevent overfull lines
\providecommand{\tightlist}{\setlength{\itemsep}{0pt}\setlength{\parskip}{0pt}}

% Redefines (sub)paragraphs to behave more like sections
\ifx\paragraph\undefined\else
\let\oldparagraph\paragraph
\renewcommand{\paragraph}[1]{\oldparagraph{#1}\mbox{}}
\fi
\ifx\subparagraph\undefined\else
\let\oldsubparagraph\subparagraph
\renewcommand{\subparagraph}[1]{\oldsubparagraph{#1}\mbox{}}
\fi



\date{2023-05-22}


\begin{document}



\maketitle

\renewcommand{\shortauthors}{Grayson et al.}


\hypertarget{introduction}{%
\section{Introduction}\label{introduction}}

A computational experiment is reproducible if another team using the
same experimental infrastructure can make a measurement that concurs
with the original. In practice, reproducers often need to manually work
with the code to see how to build necessary libraries, configure
parameters, find data, and invoke the experiment; it is not
\emph{automatic}. Automatic reproducibility is a more stringent goal,
but working towards it would benefit the community.

This work discusses a machine-readable language for specifying how to
execute a computational experiment. It is not enough for the language to
merely contain a run command in a heap of other commands; e.g., a
Makefile which defines a rule for executing the experiment alongside
rules for compiling intermediate pieces is not sufficient because there
is no machine-readable way to know which of the Make rules executes the
experiment. Being able to automatically identify the ``main'' command
which executes the experiment, for instance, would be very useful for
those seeking to reproduce results from past experiments or reusing
experiments to address new use cases. Moreover, from a research
perspective, having a standardized way to run many different codes at
scale would open new avenues for data mining research on reproducibility
(c.f., \cite{collberg_repeatability_2016}). We invite stakeholders to
discuss this language at
\url{https://github.com/charmoniumQ/execution-description}.

Even with workflows, correctly invoking the experiment is still not
automatic. In a recent study, more than 70\% of workflows do not work
out-of-the-box \cite{grayson_automatic_2023}; for instance, they might
require the user to specify data or configure parameters for their
use-case. While flexibility is desirable, it should not preclude default
invocation in a standard location for testing purposes. For example, the
Snakemake workflow engine has a standard\footnote{See Snakemake Catalog
  rules for inclusion
  \url{https://snakemake.github.io/snakemake-workflow-catalog/?rules=true}}
for documenting the required arguments of its workflows, this standard
does not have a place to put an example invocation\footnote{See this
  discussion on GitHub
  \url{https://github.com/snakemake-workflows/dna-seq-varlociraptor/pull/204\#issuecomment-1432876029}}.

\hypertarget{towards-a-standard-for-automatic-reproducibility}{%
\section{Towards a Standard for Automatic
Reproducibility}\label{towards-a-standard-for-automatic-reproducibility}}

There is a diverse range of solutions for expressing how to run code,
including bash scripts, environment management specifications (e.g.,
Spack, Nix, Python Virtualenv), continuous integration scripts,
workflows, and container specifications. In our research on the
reproducibility of scientific codes, as we scale up our studies to
include many different codes, keeping track of how to execute each one
becomes very complicated. Moreover, when a code fails to run or deliver
reproducible results, it is difficult to assess whether there is a fault
with the code or whether we did not invoke the code as intended. While
we do not expect (or recommend) that the scientific software community
converge on a single solution for executing codes, we see value in
having a standard way of documenting how to run each code that could
hand off to the user's tool of choice.

One could implement such a language using linked-data on the semantic
web. Defining the language in linked data lets us seamlessly link to
existing resources described by existing ontologies such as RO-crate
\cite{soiland-reyes_packaging_2022}, Dublin Core metadata terms
\cite{weibel_dublin_2000}, Description of a Project
\cite{wilder-james_description_2017}, nanopublications
\cite{groth_anatomy_2010}, Citation Typing Ontology
\cite{shotton_cito_2010}, and Document Components Ontology
\cite{constantin_document_2016}.

At the most basic level, the automatic reproducibility specification
should allow one to specify available commands and a fixed string
describing their purpose, e.g., run make to compile underlying libraries
and run main.py to generate figures (see \texttt{\#make} Appendix I).
The strings could be something like ``compile'', ``run'', or
``make-figures'', which would be used the same way by multiple projects.
However, the language should go beyond fixed-strings.

The language should allow users to link code directly to claims made in
publications (see \texttt{\#links-to-pub} in Appendix I). With such a
specification, any person (or program) should be able to execute the
experiments which generate figures or claims in an accompanying paper.
For example, the CiTO vocabulary \cite{shotton_cito_2010} can encode to
how the result is used as evidence in a specific publication. These
references could connect to other references of the same publication on
the semantic web.

The description can be even more granular than a publication or a fixed
string. One could use the DoCO vocabulary
\cite{constantin_document_2016} to point to specific elements (e.g.,
figures, tables, or sentences) within a document. Alternatively, one
could reference specific scientific published or unpublished claims
using the Nanopublication vocabulary \cite{groth_anatomy_2010} (see
\texttt{\#links-to-fig}, \texttt{\#defines-nanopub}, and
\texttt{\#links-to-nanopub} in Appendix I).

RO-crate \cite{soiland-reyes_wf4ever_2013} has terms for describing
dependencies between steps, which can be used to encode dependent steps
or specify the computational environment (see \texttt{\#make-data} and
\texttt{\#plot-figures} in Appendix I). The purpose of encoding
dependencies is not to usurp the build-system or workflow engine, which
both already handle task dependencies; if the experiment already uses a
workflow, then the specification should invoke that. The purpose of task
dependencies in the specification is for projects which do not use a
workflow engine, or a task that installs the desired workflow engine.

Such a specification could also set bounds on the experiment's
parameters, such as the range of valid values or a list of toggleable
parameters. See node \texttt{\#example-of-parameters} in Appendix I for
example. This parameter metadata would enable downstream automated
experiments like parameter-space search studies, multi-fidelity
uncertainty quantification, and outcome-preserving input minimization.

\hypertarget{getting-adoption}{%
\section{Getting Adoption}\label{getting-adoption}}

The most useful part of the specification would need \emph{some} human
input to create; it is not just specifying tasks but what those steps
do. However, we can reduce the manual effort needed to write the
specification.

Workflow engines could assist in generating this. Workflow engines know
all the computational steps, inputs, outputs, and parameters. Then it
could prompt the user with high-level questions (e.g., ``What
publication is this part of''?) and generate the appropriate
specification.

If the experiment does not use a workflow engine, but someone who can
run the experiment is available, an interactive shell session can
capture and write the specification. The user would invoke a shell that
records every command, its exit status, its read-files, and its
write-files (using syscall interposition); The user would run their code
as usual, and after finishing, the shell would assemble the necessary
computational steps and prompt the user for high-level questions.

As a last resort, if one finds a publication linking to a specific
repository, one can try to guess the main command. This approach is the
current state-of-the-art for large-scale reproduction studies, except a
standardized language would allow some large-scale reproduction studies
to inform future large-scale reproduction studies on what they did to
execute this repository. Computational scientists at least had an
opportunity to influence how to invoke their code in large-scale
reproduction studies. The lack of opportunity for input was a frequent
response of scientists to Collberg and Proebsting\footnote{See ``Author
  Comments'' in
  \url{http://reproducibility.cs.arizona.edu/v2/index.html}. The authors
  of publications whose labels are BarowyCBM12, BarthePB12,
  HolewinskiRRFPRS12, and others responded to Collberg and Proebsting
  (paraphrasing), ``it would have worked; you just didn't invoke the
  right commands.''}.

Computational scientists could benefit from creating these automated
reproducibility specifications because large-scale reproduction studies
like Collberg and Proebsting \cite{collberg_repeatability_2016}, Zhao et
al.~\cite{zhao_why_2012}, and others serve as free testing and
reproduction of their results.

Ideally, the reproduction specification would be placed in the same
location as the computational experiment, often a GitHub repository, so
developers can maintain it alongside the code. In cases where the
authors of the GitHub repository are not cooperative, one can instead
put reproduction specifications in a repository that holds reproduction
specifications from the community, a ``reproducibility library''. Users
seeking to reproduce a repository would invoke a tool that looks for an
automatic reproducibility specification in the source code repository,
in a list of reproducibility libraries, and if none exists there, falls
back on heuristic to guess how to reproduce the experiment. If the
fallback succeeds, the tool can upload all its steps to a
reproducibility library.

Meanwhile, conferences and publishers could promote such standard
specifications as part of reproducibility requirements for publishing.
Currently, to get an artifact evaluation badge, computational scientists
would have to write a natural language description of the software
environment, what the commands are, how to run them, and where the data
end up; meanwhile, an artifact evaluator has to read, interpret, and
execute their description by hand. An execution description could make
this nearly automatic; if an execution description exists, the artifact
evaluator uses an executor which understands the language and runs all
of the commands that reference the manuscript in their \texttt{purpose}
tag.

\hypertarget{conclusion}{%
\section{Conclusion}\label{conclusion}}

Developing common standards for specifying how to run computational
experiments would benefit the scientific community. It presents a
compromise where different teams can implement their codes however they
see fit while enabling others to run them easily. This specification
would lead to greater productivity in the (re)use of scientific
experiments, empower developers to build tools that leverage those
common specifications, and enable software engineering researchers to
study reproducibility at scale. Help us form a consensus around a
particular language by contributing to
\url{https://github.com/charmoniumQ/execution-description}.

\hypertarget{appendix-i-example-document}{%
\section{Appendix I: Example
document}\label{appendix-i-example-document}}

The following language sample is not the final proposal for the complete
vocabulary; the peer-review process is not well-suited to iterate on
technical details. The point of this article is to argue that the
community should spend effort developing this vocabulary.

\small

\begin{Shaded}
\begin{Highlighting}[]
\FunctionTok{\textless{}?xml}\OtherTok{ version=}\StringTok{"1.0"}\OtherTok{ encoding=}\StringTok{"utf{-}8"}\FunctionTok{?\textgreater{}}
\CommentTok{\textless{}!{-}{-}}
\CommentTok{RDF can be serialized as XML, JSON, or triples; backend RDF parsers don\textquotesingle{}t care.}
\CommentTok{We chose XML because it might be more familiar to readers.}
\CommentTok{{-}{-}\textgreater{}}

\CommentTok{\textless{}!{-}{-}}
\CommentTok{The following tag imports several other vocabularies behind a namespace.}
\CommentTok{E.g., \textasciigrave{}rdf:type\textasciigrave{} refers to \textasciigrave{}type\textasciigrave{} in the \textasciigrave{}rdf\textasciigrave{} namespace, which resolves to:}
\CommentTok{http://www.w3.org/1999/02/22{-}rdf{-}syntax{-}ns\#rdftype}
\CommentTok{Elements with no namespace are resolved within the default namespace,}
\CommentTok{which is our proposed execution{-}description vocabulary, http://example.org/execution{-}description/1.0.}
\CommentTok{{-}{-}\textgreater{}}

\NormalTok{\textless{}}\KeywordTok{rdf:RDF}\OtherTok{ xmlns:rdf=}\StringTok{"http://www.w3.org/1999/02/22{-}rdf{-}syntax{-}ns\#"}
\OtherTok{         xmlns:rdfs=}\StringTok{"http://www.w3.org/2000/01/rdf{-}schema\#"}
\OtherTok{         xmlns:dc=}\StringTok{"http://purl.org/dc/elements/1.1/"}
\OtherTok{         xmlns:wikibase=}\StringTok{"http://wikiba.se/ontology\#"}
\OtherTok{         xmlns:cito=}\StringTok{"http://purl.org/spar/cito"}
\OtherTok{         xmlns:doco=}\StringTok{"http://purl.org/spar/doco/2015{-}07{-}03"}
\OtherTok{         xmlns:prov=}\StringTok{"http://www.w3.org/TR/2013/PR{-}prov{-}o{-}20130312/"}
\OtherTok{         xmlns:wfdesc=}\StringTok{"http://purl.org/wf4ever/wfdesc\#"}
\OtherTok{         xml:lang=}\StringTok{"en"}
\NormalTok{         \textgreater{}}

  \CommentTok{\textless{}!{-}{-}}
\CommentTok{  Here, we list some relevant commands, and how they relate to the artifact.}
\CommentTok{  {-}{-}\textgreater{}}
\NormalTok{  \textless{}}\KeywordTok{process}\OtherTok{ rdf:about=}\StringTok{"\#make"}\NormalTok{\textgreater{}}
    \CommentTok{\textless{}!{-}{-} The following would get run by the UNIX shell. {-}{-}\textgreater{}}
\NormalTok{    \textless{}}\KeywordTok{command}\NormalTok{\textgreater{}make libs\textless{}/}\KeywordTok{command}\NormalTok{\textgreater{}}
    \CommentTok{\textless{}!{-}{-} Here is a string representing the purpose. {-}{-}\textgreater{}}
\NormalTok{    \textless{}}\KeywordTok{purpose}\NormalTok{\textgreater{}compiles libraries\textless{}/}\KeywordTok{purpose}\NormalTok{\textgreater{}}
\NormalTok{  \textless{}/}\KeywordTok{process}\NormalTok{\textgreater{}}

  \CommentTok{\textless{}!{-}{-}}
\CommentTok{  Here, we make a process that depends on a previous process using wfdesc.}
\CommentTok{  {-}{-}\textgreater{}}
\NormalTok{  \textless{}}\KeywordTok{process}\OtherTok{ rdf:about=}\StringTok{"\#make{-}data"}\NormalTok{\textgreater{}}
\NormalTok{    \textless{}}\KeywordTok{command}\NormalTok{\textgreater{}python3 make\_data.py\textless{}/}\KeywordTok{command}\NormalTok{\textgreater{}}
\NormalTok{    \textless{}}\KeywordTok{purpose}\NormalTok{\textgreater{}makes data\textless{}/}\KeywordTok{purpose}\NormalTok{\textgreater{}}
\NormalTok{  \textless{}/}\KeywordTok{process}\NormalTok{\textgreater{}}
\NormalTok{  \textless{}}\KeywordTok{process}\OtherTok{ rdf:about=}\StringTok{"\#plot{-}figures"}\NormalTok{\textgreater{}}
\NormalTok{    \textless{}}\KeywordTok{command}\NormalTok{\textgreater{}python3 figures.py\textless{}/}\KeywordTok{command}\NormalTok{\textgreater{}}
\NormalTok{    \textless{}}\KeywordTok{purpose}\NormalTok{\textgreater{}plot figures\textless{}/}\KeywordTok{purpose}\NormalTok{\textgreater{}}
\NormalTok{    \textless{}}\KeywordTok{dependsOn}\OtherTok{ rdf:resource=}\StringTok{"\#make{-}data"}\NormalTok{ /\textgreater{}}
    \CommentTok{\textless{}!{-}{-}}
\CommentTok{    The \# is not a typo; the rdf:about becomes a URL fragment in the current document.}
\CommentTok{    This means one can access a computational step in another document here,}
\CommentTok{    like "https://example.com/software{-}experiment{-}23\#make{-}data".}
\CommentTok{    {-}{-}\textgreater{}}
\NormalTok{  \textless{}/}\KeywordTok{process}\NormalTok{\textgreater{}}
  \CommentTok{\textless{}!{-}{-} Users may choose the more complex wfdesc vocabulary if they wish. {-}{-}\textgreater{}}

  \CommentTok{\textless{}!{-}{-}}
\CommentTok{  Links to a publication.}
\CommentTok{  The publisher may or may not host a linked{-}data description of the documenta at this URL.}
\CommentTok{  The purpose of the URL is to unambiguously name the document.}
\CommentTok{  We need the rdf:Description to reference an external resource.}
\CommentTok{  {-}{-}\textgreater{}}
\NormalTok{  \textless{}}\KeywordTok{process}\OtherTok{ rdf:about=}\StringTok{"links{-}to{-}pub"}\NormalTok{\textgreater{}}
\NormalTok{    \textless{}}\KeywordTok{command}\NormalTok{\textgreater{}make all\textless{}/}\KeywordTok{command}\NormalTok{\textgreater{}}
\NormalTok{    \textless{}}\KeywordTok{purpose}\NormalTok{\textgreater{}}
\NormalTok{      \textless{}}\KeywordTok{rdf:Description}\NormalTok{\textgreater{}}
\NormalTok{        \textless{}}\KeywordTok{cito:isCitedAsEvidenceBy}\OtherTok{ rdf:resource=}\StringTok{"https://doi.org/10.1234/123456789"}\NormalTok{ /\textgreater{}}
\NormalTok{      \textless{}/}\KeywordTok{rdf:Description}\NormalTok{\textgreater{}}
\NormalTok{    \textless{}/}\KeywordTok{purpose}\NormalTok{\textgreater{}}

  \CommentTok{\textless{}!{-}{-} Links to a specific figure within a publication {-}{-}\textgreater{}}
\NormalTok{  \textless{}}\KeywordTok{process}\OtherTok{ rdf:about=}\StringTok{"links{-}to{-}fig"}\NormalTok{\textgreater{}}
\NormalTok{    \textless{}}\KeywordTok{command}\NormalTok{\textgreater{}make all\textless{}/}\KeywordTok{command}\NormalTok{\textgreater{}}
\NormalTok{    \textless{}}\KeywordTok{purpose}\NormalTok{\textgreater{}}
\NormalTok{      \textless{}}\KeywordTok{prov:generated}\NormalTok{\textgreater{}}
\NormalTok{        \textless{}}\KeywordTok{doco:figure}\NormalTok{\textgreater{}}
\NormalTok{          \textless{}}\KeywordTok{rdf:Description}\NormalTok{\textgreater{}}
\NormalTok{            \textless{}}\KeywordTok{dc:title}\NormalTok{\textgreater{}Figure 2b\textless{}/}\KeywordTok{dc:title}\NormalTok{\textgreater{}}
\NormalTok{            \textless{}}\KeywordTok{dc:isPartOf}\OtherTok{ rdf:resource=}\StringTok{"https://doi.org/10.1234/123456789"}\NormalTok{ /\textgreater{}}
\NormalTok{          \textless{}/}\KeywordTok{rdf:Description}\NormalTok{\textgreater{}}
\NormalTok{        \textless{}/}\KeywordTok{doco:figure}\NormalTok{\textgreater{}}
\NormalTok{      \textless{}/}\KeywordTok{prov:generated}\NormalTok{\textgreater{}}
\NormalTok{    \textless{}/}\KeywordTok{purpose}\NormalTok{\textgreater{}}

  \CommentTok{\textless{}!{-}{-}}
\CommentTok{  Describes an abstract nanopublication claim that this experiment supports.}
\CommentTok{  This one will say: "this experiment supports the claim that malaria is spread by mosquitoes"}
\CommentTok{  {-}{-}\textgreater{}}
\NormalTok{  \textless{}}\KeywordTok{process}\OtherTok{ rdf:about=}\StringTok{"defines{-}nanopub"}\NormalTok{\textgreater{}}
\NormalTok{    \textless{}}\KeywordTok{command}\NormalTok{\textgreater{}make all\textless{}/}\KeywordTok{command}\NormalTok{\textgreater{}}
\NormalTok{    \textless{}}\KeywordTok{purpose}\NormalTok{\textgreater{}}
\NormalTok{      \textless{}}\KeywordTok{cito:supports}\NormalTok{\textgreater{}}
        \CommentTok{\textless{}!{-}{-}}
\CommentTok{        We will use Wikidata here.}
\CommentTok{        They have catalogued many real{-}world objects and concepts as linked{-}data objects.}
\CommentTok{        {-}{-}\textgreater{}}
\NormalTok{        \textless{}}\KeywordTok{wikibase:Statement}\NormalTok{\textgreater{}}
\NormalTok{          \textless{}}\KeywordTok{rdf:Description}\NormalTok{\textgreater{}}
            \CommentTok{\textless{}!{-}{-} Q12156 refers to malaria {-}{-}\textgreater{}}
\NormalTok{            \textless{}}\KeywordTok{subject}\OtherTok{ rdf:resource=}\StringTok{"https://www.wikidata.org/entity/Q12156"}\NormalTok{ /\textgreater{}}
            \CommentTok{\textless{}!{-}{-} P1060 refers to disease transmission process (read: "is transmitted by") {-}{-}\textgreater{}}
\NormalTok{            \textless{}}\KeywordTok{predicate}\OtherTok{ rdf:resource=}\StringTok{"http://www.wikidata.org/prop/P1060"}\NormalTok{ /\textgreater{}}
            \CommentTok{\textless{}!{-}{-} Q15304532 refers to mosquitoes {-}{-}\textgreater{}}
\NormalTok{            \textless{}}\KeywordTok{object}\OtherTok{ rdf:resource=}\StringTok{"https://www.wikidata.org/entity/Q15304532"}\NormalTok{ /\textgreater{}}
\NormalTok{          \textless{}/}\KeywordTok{rdf:Description}\NormalTok{\textgreater{}}
\NormalTok{        \textless{}/}\KeywordTok{wikibase:Statement}\NormalTok{\textgreater{}}
\NormalTok{      \textless{}/}\KeywordTok{cito:supports}\NormalTok{\textgreater{}}
\NormalTok{    \textless{}/}\KeywordTok{purpose}\NormalTok{\textgreater{}}

    \CommentTok{\textless{}!{-}{-}}
\CommentTok{    Alternatively, the nanopublication claim will live somewhere else.}
\CommentTok{    Linked data lets us seamlessly reference other documents.}
\CommentTok{    {-}{-}\textgreater{}}
\NormalTok{    \textless{}}\KeywordTok{purpose}\OtherTok{ rdf:about=}\StringTok{"links{-}to{-}nanopub"}\NormalTok{\textgreater{}}
\NormalTok{      \textless{}}\KeywordTok{rdf:Description}\NormalTok{\textgreater{}}
\NormalTok{        \textless{}}\KeywordTok{cito:supports}\OtherTok{ rdf:resource=}\StringTok{"https://example.com/article24\#claim31"}\NormalTok{ /\textgreater{}}
\NormalTok{      \textless{}/}\KeywordTok{rdf:Description}\NormalTok{\textgreater{}}
\NormalTok{    \textless{}/}\KeywordTok{purpose}\NormalTok{\textgreater{}}
\NormalTok{  \textless{}/}\KeywordTok{process}\NormalTok{\textgreater{}}

  \CommentTok{\textless{}!{-}{-} Here, we add parameters to the command {-}{-}\textgreater{}}
\NormalTok{  \textless{}}\KeywordTok{process}\OtherTok{ rdf:label=}\StringTok{"example{-}of{-}parameters"}\NormalTok{\textgreater{}}
    \CommentTok{\textless{}!{-}{-} These might be template filled like so: {-}{-}\textgreater{}}
\NormalTok{    \textless{}}\KeywordTok{command}\NormalTok{\textgreater{}./generate $\{max\_resolution\} $\{rounds\}\textless{}/}\KeywordTok{command}\NormalTok{\textgreater{}}
\NormalTok{    \textless{}}\KeywordTok{wfdesc:Parameter}\OtherTok{ rdfs:label=}\StringTok{"max\_resolution"}\NormalTok{ /\textgreater{}}
\NormalTok{  \textless{}/}\KeywordTok{process}\NormalTok{\textgreater{}}

\NormalTok{\textless{}/}\KeywordTok{rdf:RDF}\NormalTok{\textgreater{}}
\end{Highlighting}
\end{Shaded}

\normalsize

The above RDF/XML can be validated with Python and rdflib:

\begin{Shaded}
\begin{Highlighting}[]
\OperatorTok{\textgreater{}\textgreater{}\textgreater{}} \ImportTok{import}\NormalTok{ rdflib}
\OperatorTok{\textgreater{}\textgreater{}\textgreater{}}\NormalTok{ g }\OperatorTok{=}\NormalTok{ rdflib.Graph().parse(}\StringTok{"test.xml"}\NormalTok{)}
\OperatorTok{\textgreater{}\textgreater{}\textgreater{}} \CommentTok{\# Now we can iterate over the triples contained in this RDF graph}
\OperatorTok{\textgreater{}\textgreater{}\textgreater{}} \CommentTok{\# Note that "anonymous nodes" will appear as rdflib.term.BNode(\textquotesingle{}...\textquotesingle{})}
\OperatorTok{\textgreater{}\textgreater{}\textgreater{}} \BuiltInTok{list}\NormalTok{(g)[:}\DecValTok{5}\NormalTok{]}
\NormalTok{[(rdflib.term.BNode(}\StringTok{\textquotesingle{}N979c272652c948f48598caa65eaf02da\textquotesingle{}}\NormalTok{),}
\NormalTok{  rdflib.term.URIRef(}\StringTok{\textquotesingle{}http://www.w3.org/1999/02/22{-}rdf{-}syntax{-}ns\#type\textquotesingle{}}\NormalTok{),}
\NormalTok{  rdflib.term.URIRef(}\StringTok{\textquotesingle{}http://www.w3.org/TR/2013/PR{-}prov{-}o{-}20130312/generated\textquotesingle{}}\NormalTok{)),}
\NormalTok{ (rdflib.term.URIRef(}\StringTok{\textquotesingle{}file:///home/sam/box/execution{-}description/se4rs/test.xml\#plot{-}figures\textquotesingle{}}\NormalTok{),}
\NormalTok{  rdflib.term.URIRef(}\StringTok{\textquotesingle{}file:///.../purpose\textquotesingle{}}\NormalTok{),}
\NormalTok{  rdflib.term.Literal(}\StringTok{\textquotesingle{}plot figures\textquotesingle{}}\NormalTok{, lang}\OperatorTok{=}\StringTok{\textquotesingle{}en\textquotesingle{}}\NormalTok{)),}
\NormalTok{ (rdflib.term.BNode(}\StringTok{\textquotesingle{}N979c272652c948f48598caa65eaf02da\textquotesingle{}}\NormalTok{),}
\NormalTok{  rdflib.term.URIRef(}\StringTok{\textquotesingle{}http://purl.org/spar/doco/2015{-}07{-}03figure\textquotesingle{}}\NormalTok{),}
\NormalTok{  rdflib.term.BNode(}\StringTok{\textquotesingle{}Ned5bd1d9a83b48bfa0798f2f1e296db7\textquotesingle{}}\NormalTok{)),}
\NormalTok{ (rdflib.term.BNode(}\StringTok{\textquotesingle{}Nc4f1068252194a4d90b91a02f3860cf7\textquotesingle{}}\NormalTok{),}
\NormalTok{  rdflib.term.URIRef(}\StringTok{\textquotesingle{}http://wikiba.se/ontology\#Statement\textquotesingle{}}\NormalTok{),}
\NormalTok{  rdflib.term.BNode(}\StringTok{\textquotesingle{}Nce17a7a5920846788169b713dd655c97\textquotesingle{}}\NormalTok{)),}
\NormalTok{ (rdflib.term.BNode(}\StringTok{\textquotesingle{}N889f577571ab4c67bc063a0d032eb5cf\textquotesingle{}}\NormalTok{),}
\NormalTok{  rdflib.term.URIRef(}\StringTok{\textquotesingle{}file:///.../purpose\textquotesingle{}}\NormalTok{),}
\NormalTok{  rdflib.term.BNode(}\StringTok{\textquotesingle{}Nc4f1068252194a4d90b91a02f3860cf7\textquotesingle{}}\NormalTok{))]}
\end{Highlighting}
\end{Shaded}




%%\printbibliography
%
\bibliographystyle{ACM-Reference-Format}
\bibliography{sams-zotero-export,manual}

\end{document}
